\documentclass[a4paper,11pt]{article}  
\usepackage{graphicx}
\usepackage{hyperref}


\begin{document}
%\begin{center}
 %   \Large{SC'17 workshop proposal}\\~\\
%    \large{\textbf{4th International Workshop on HPC User Support Tools (HUST-17)}}\\~\\
 %   \normalsize{Ralph C. Bording (Pawsey Supercomputing Centre), Todd Gamblin (LLNL),\\
 %   Olli-Pekka Lehto(Jump Trading,LLC)}
%\end{center}

\section*{Abstract}
\label{sec:abstract}

Supercomputing centers exist to drive scientific discovery by supporting researchers in 
computational science fields.  To make users more productive in the complex HPC
environment, HPC centers employ user support teams.  These teams
serve many roles, from setting up accounts, to consulting on math libraries and code
optimization, to managing HPC software stacks.
Often, support teams struggle to adequately support scientists.
HPC environments are extremely complex, and combined with
the complexity of multi-user installations, exotic hardware, and maintaining
research software, supporting HPC users can be extremely demanding.

With the fourth HUST workshop, we will continue to provide a necessary forum for 
system administrators, user support team members, tool developers, policy makers and
end users.  We will provide a forum to discuss support issues and we will
provide a publication venue for current support developments.  Best practices,
user support tools, and any ideas to streamline user support at supercomputing
centers are in scope.

%\paragraph{Topic area:} Administration tools, User support
%\paragraph{Keywords:} user support, tools, best practices, Docker, big data, Hadoop and Spark clusters, machine learning, collaboration, community

\end{document}

\

