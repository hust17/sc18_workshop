\documentclass[a4paper,10pt]{article}  
\usepackage{graphicx}
\usepackage{hyperref}
\setlength{\textwidth}{6.5in}
\setlength{\textheight}{8.6in}
\setlength{\topmargin}{0.0in}
\setlength{\headheight}{0.0in}
\setlength{\oddsidemargin}{0.0in}
\setlength{\evensidemargin}{0.0in}

%\usepackage{savetrees}
\usepackage{wrapfig}
\usepackage{times}
\usepackage{fullpage}
\usepackage{multicol}

\usepackage{enumitem}
\setlist{nosep} % or \setlist{noitemsep} to leave space around whole list

% stolen from moderncv
\newcommand*{\hintfont}{\small\sffamily}
\newlength{\hintscolumnwidth}
\setlength\hintscolumnwidth{1,79cm}
\newlength{\separatorcolumnwidth}
\setlength{\separatorcolumnwidth}{0.025\textwidth}
\newlength{\maincolumnwidth}
  \setlength{\maincolumnwidth}{\textwidth}%
  \addtolength{\maincolumnwidth}{-\separatorcolumnwidth}%
  \addtolength{\maincolumnwidth}{-\hintscolumnwidth}%
  
\newcommand*{\cvline}[3][.25em]{%
\begin{tabular}{@{}p{\hintscolumnwidth}@{\hspace{\separatorcolumnwidth}}p{\maincolumnwidth}@{}}%
 	  \raggedleft\hintfont{#2} &{#3}%
  \end{tabular}\\[#1]}
  
% usage: \cventry{years}{degree/job title}{institution/employer}{localization}{optionnal: grade/...}{optional: comment/job description}
\newcommand*{\cventry}[6]{%
  \cvline{#1}{%
    {\bfseries#2}%
    \ifx#3\else{, {\slshape#3}}\fi%
    \ifx#4\else{, #4}\fi%
    \ifx#5\else{, #5}\fi%
    .%
    \ifx#6\else{\newline{}\begin{minipage}[t]{\linewidth}\small#6\end{minipage}}\fi
    }}%  


\newcommand{\ignore}[1]{}

\newif\ifremark
\long\def\remark#1{
\ifremark%
        \begingroup%
        \dimen0=\columnwidth
        \advance\dimen0 by -1in%
        \setbox0=\hbox{\parbox[b]{\dimen0}{\protect\em #1}}
        \dimen1=\ht0\advance\dimen1 by 2pt%
        \dimen2=\dp0\advance\dimen2 by 2pt%
        \vskip 0.25pt%
        \hbox to \columnwidth{%
                \vrule height\dimen1 width 3pt depth\dimen2%
                \hss\copy0\hss%
                \vrule height\dimen1 width 3pt depth\dimen2%
        }%
        \endgroup%
\fi}

%%\remarktrue
\remarkfalse

\newenvironment{myenumerate}
{ \begin{enumerate}
    \setlength{\itemsep}{0pt}
    \setlength{\parskip}{0pt}
    \setlength{\parsep}{0pt}     }
{ \end{enumerate}                  } 

\begin{document}
\begin{center}
    \Large{SC'17 workshop proposal}\\~\\
    \large{\textbf{4th International Workshop on HPC User Support Tools (HUST-170)}}\\~\\
    \normalsize{Ralph C. Bording (Pawsey Supercomputing Centre), Todd Gamblin (LLNL),\\
    Olli-Pekka Lehto (Jump Trading, LLC)}
\end{center}

\section*{Abstract}
\label{sec:abstract}

Supercomputing centers exist to drive scientific discovery by supporting researchers in 
computational science fields.  To make users more productive in the complex HPC
environment, HPC centers employ user support teams.  These teams
serve many roles, from setting up accounts, to consulting on math libraries and code
optimization, to managing HPC software stacks.
Often, teams struggle to adequately support scientists.
HPC environments are very complex, and combined with
the complexity of multi-user installations, exotic hardware, and
research software, supporting HPC users can be demanding.

With the fourth HUST workshop, we will continue to provide a necessary forum for 
system administrators, support team members, developers, policy makers and
end users.  We will provide forum to discuss support issues and we will
provide a publication venue for current support developments.  Best practices,
user support tools, and any ideas to streamline user support at supercomputing
centers are in scope.

%\paragraph{Topic area:} Administration tools, User support
%\paragraph{Keywords:} user support, tools, best practices, collaboration, community

\section*{Detailed Description}
\label{sec:detailed}

\subsection*{Motivation}
\label{sec:motivation}

Every high-performance computing (HPC) site exists to support its users,
be they academic, governmental or industrial.  Users run complex applications
that stress the limits of HPC resources and software environments.
Users come from a wide variety of fields,
and typically they can only maximise their productivity with the help of
committed user support teams.  Support entails a wide range
of tasks, e.g., installing and building (scientific) software packages; providing
performance analysis tools; user training, education, and documentation; testing
systems, and providing frequent system and OS upgrades.

Over the past few years, we have spoken with members of many HPC support teams,
and we found that many sites struggle with providing adequate user support.  
There are two reasons: (i) lack of funding and therefore manpower, and
(ii) lack of good tools to automate ubiquitous, labor-intensive tasks.
%
Given the limited manpower, and given the commonality of tasks across both large
and small HPC sites, it surprised us that there was not more collaboration among
HPC centers.  This would allow effort to be leveraged across different HPC sites.
For this reason, we started the HUST workshop at SC14.                                                              

The HUST workshop at SC provides the forum for the 
dissemination of knowledge about best practices and tools for HPC user support teams. 
Our goal now is to continue the development of a community to aid HPC teams 
with their user support efforts and to publish peer-reviewed papers through the ACM Digital Library \cite{H14} \cite{H15} \cite{H16}
. 

The HUST-16 workshop was moved to Sunday morning at SC16 and we saw an increase in the attendees up to 95 compared to 75 at HUST-15 .  The HUST-16 workshop again scored well above the conference average in 'Technical Content', 'Quality of Presentations','Overall Value' and 'Workshop Program', and based on the attendees comments they found the workshop and the presentations of value as HPC practictioners. 

We anticipate that HUST-17 will continue this successful tradition.

\subsection*{Scope}
\label{sec:scope}

To ease the burden of user support, HPC center staff must adopt collaborative
methodologies and practices already used by the scientists and researchers
they support: continual sharing, collaboration, and evaluation of methods and tools.
The scope of this workshop is to provide a forum to bring together HPC user support
teams, allowing them to share their ideas and experiences across a range of topics
directly related to improving HPC user support. This includes, but is not limited to, 
raising awareness and increasing uptake of existing production and research tools.
As a community we should be able to further improve these tools, fix their current
shortcomings and evangelise their existence in a much more efficient manner. HPC user 
support teams often work in isolation, and this workshop will allow them to benefit 
from knowledge sharing and to reduce duplication of effort across sites.
This in turn allows the community as a whole to benefit from these tools and improve
the user experience on HPC systems, which hopefully results in maximising the 
scientific output of each HPC center.

\vspace{1em}
\noindent
Topics of interest include, but are not limited to:
\vspace{1em}
\begin{itemize}
    \item \textbf{Defining and customising the user environment}\\
    HPC system environments are complex, dynamic, and customised
    heavily.  We will discuss automatic, flexible customisation of the user
    environment with environment modules and other tools.

    \item \textbf{Software build and installation tools}\\
    Software build and installation tools provide automatic configuration, 
    compilation and installation of
    scientific software -- this is far from trivial to do correctly,
    especially when reproducibility of built packages is desired. This includes
    tools that simplify the installation of new scientific software packages
    with many versions and configurations
%    and/or update existing ones -- without removal of formerly installed versions --
%    to meet the researcher's needs.
	It also comprises tools for allowing to
    quickly install a complex software stack onto new systems.
%    Additionally,
%    they could provide a mechanism for rebuilding a software stack as part of any
%    disaster recovery policy. We have experience with such tools, but we would
%    like to hear from other HPC sites what they use, how they tackle these
%    complex and time-consuming issues, etc.


    \item \textbf{Workflow and pipeline tools }\\
    Quite often, researchers manage their workflow using scripts in
    Bash, Python, Perl, etc. There are many workflow tools available that might
    be a better fit for increasing user productivity. Such tools typically allow managing
    complex data analysis work, automating parameter sweep experiments, provide portals
    or GUIs through which users can submit their jobs on the HPC infrastructure, etc.

	\item \textbf{Collaborative development tools}\\
	HPC centers have traditionally provided fast, well-tuned hardware, a shell
	console, and not much more.  Modern development environment make use of wikis, 
	bug trackers, repositories, build and test software, and other hosted software
	solutions.  These can be difficult to deploy when there are many users,
	particularly if there are security concerns.  We welcome all papers on enhancing
	the HPC development environment with modern HPC web tools.
	
%    \item \textbf{Novel environments: cloud, etc.}\\
%    Recently, we observed the advent of HPC in a cloud environment. Tools, best practices and 
%    applications that allow managing a cloud-based work-flow fit this part of the scope.

    \item \textbf{Supporting Hadoop and Spark clusters for \emph{Machine Learning} and \emph{Big Data}}\\
    Though we want to avoid placing too much emphasis on Big Data as such, 
    experiments involving huge amounts of data do pose challenges to user
    support teams, either to conduct on existing infrastructure or because new 
    dedicated infrastructure needs to be deployed. We expect tools that
    facilitate Hadoop or Spark workflows to become common in HPC. We are 
    interested in how centers can efficiently maintain two stacks of software,
    for Big Data using Machine Learning applications and for more traditional HPC.

%    \item \textbf{Establishing baseline configuration efforts for HPC}\\
%    Some sites have a single system in production at any point in time, others
%    have multiple (maybe smaller) clusters. Yet it is often desirable to establish 
%    a common baseline configuration for any site, to ease migration to a new system and
%    to reduce the workload placed on system administration teams -- dealing
%    with multiple different environments can be quite challenging and time consuming.


    \item \textbf{System testing and monitoring}\\
    Tools that help to test HPC systems and to optimise the performance of an HPC center
    as a whole are within scope.  This could include continuous monitoring or
    performance measurement tools, or test suites used regularly to establish baseline 
    cluster performance.
%    No paper without data, as the saying goes. Yet this requires careful thought
%    on the side of both the researcher and the support team. This sub-scope comprises 
%    tools that help storing data sets, code, etc. for the purpose of validating
%    ongoing research, for performing system level testing, etc.
\end{itemize}


\newpage
\subsection*{Format}
\label{format}

The format of the workshop will be fairly straightforward. We envision the following the schedule from SC16:

\begin{description}
    \item[9:00] introduction and survey announcement (5 minutes)
    \item[9:05] paper session 1 (2 papers, 15-20 minutes presentations + 5-10 minutes questions)
    \item[10:05] break (20 minutes)
    \item[10:25] paper session 2 (3 papers, 15-20 minutes presentations + 5-10 minutes questions)
    \item[11:25] panel discussion  (30 minutes)
	\item[12:30] End of workshop
\end{description}

%\begin{figure}[h!]
%    \begin{center}
%        \includegraphics[width=7cm]{timeline}
%    \end{center}
%    \caption{Proposed timeline for the workshop.}
%    \label{fig:timeline}
%\end{figure}

During the introduction, we will briefly explain the goal and the scope of the
workshop.%, and outline the survey we plan on organizing.
%
Throughout the workshop, we will ask users to fill out an interactive user support
survey. We have previously used Socrative\footnote{\url{http://socrative.com}}. 
The survey has been informative and has showed how the HPC community has changed
over time to adopt tools and methodologies. We expect this year's survey to be 
similarly edifying.

We plan to have two sequential tracks for paper presentations, with 5 presentations
in total. Depending on the topics of received submissions, these may be organized
into two sets of related talks. Each presentation will be 25 minutes long.
We will spur the discussion and encourage a strong interaction between the 
audience and the presenter, so the presentation should be limited to 15 minutes at
most, leaving 10 minutes for discussion.
%
We will end the workshop with a panel of experts from major HPC centers, and we will
discuss, in light of the presented papers, what the biggest user support issues are.
%Finally, we will end the workshop with a brief review of collected survey results and
%a panel discussion.  
We plan to have the workshop last about 3.5 hours (half-day).
A detailed timeline is provided above.


\subsection*{Expected outcome}
\label{outcome}

The main outcome of this workshop is to continue to build a community to enhance 
the knowledge, tools and best practices for aiding HPC teams with their user support efforts.
Additionally, this workshop will aid in building awareness about the current lack of
collaboration and hopefully bring in more HPC sites that wish to share their
successes and their failures. The main goal is to encourage reuse of efforts and to
learn from each other, because every  HPC site faces similar challenges at the end of the day.

Second, we again plan to publish the reviewed and accepted papers through a widely
recognized channel.  Last year we published the workshop papers in the IEEE.
We will submit a request to SIGHPC to allow the workshop proceedings to be included
in these libraries. As fallbacks, we will also consider ACM Digital Library and Lecture Notes in
Computer Science\footnote{\url{http://www.springer.com/computer/lncs}} (LNCS).

Finally, a workshop report will be written and distributed. Based on the success of the
past workshops, we plan to continue this event and as a hub for the HPC 
user support community.  Based on the increase in the number attendees who joined us last year,
and the positive feedback in the reviews and the discussion after the workshop we believe
that our goal is within reach, and that this workshop will continue to encourage 
better collaboration.


\subsection*{Advertising and soliciting submissions}
\label{advertising}

We intend to distribute the call for papers for the workshop via the usual channels, including:

\begin{itemize}
    \item Mailing lists and forums specific to HPC communities;
    \item use EasyChair's new CFP feature
    \item Social networks: Twitter, Reddit, IRC, \ldots;
    \item Promotion at various HPC events, such as the Cray User Group 2017(Seatle-USA,May 2017, ISC'17 (Frankfurt - Germany, June 2017). ICS, and other HPC Conferences.
\end{itemize}
\noindent
In addition, we will also contact relevant peers in our professional network, including the PRACE consortium, the EasyBuild, Spack and Maali build system communities, DOE National Laboratories, and other groups that include HPC sites worldwide.

\subsection*{Timeline}
\label{timeline}

We plan to adhere to the following timeline:

\begin{itemize}
    \item \textbf{Call for papers} issued and \textbf{submissions open}: end of April 2017\footnote{The exact date depends on the SIGHPC notification.}
    \item \textbf{Submission deadline}: August 18th 2017 (optional extension: August 25th 2017)
    \item \textbf{Workshop paper reviews}: by September 18th 2017
    \item \textbf{Final program committee meeting}: week of 18th September 2017
    \item \textbf{Acceptance notifications}: September 22nd 2017
    \item \textbf{Camera-ready papers}: October 13th 2017
    \item \textbf{Workshop}: At SC'17
\end{itemize}
\noindent
We will open submissions as soon a possible after acceptance notification, taking into account time required for confirmation of SIGHPC cooperation status (4 weeks). With acceptance notification planned for late March 2017, we should be able to open submissions \textbf{end of April 2017}.
%
The submission deadline is tentatively \textbf{August 18th 2017}, optionally extended to August 19th in case
we are not satisfied with the number of submissions. We hope to get at least 10 submissions again.
%
%PC members should review their papers by \textbf{September 18th 2017},
%and the final PC meeting will take place in the \textbf{the week of 18th September 2017}.
%
%This should allow us to notify authors on \textbf{September 22nd 2017}.  We will require camera-ready workshop papers to be submitted by \textbf{October 13th 2017}, in time for the October 15th SIGHPC deadline.
%
The workshop should take place on \textbf{November 12th, 13th or 17th, 2017}.

\section*{Potential program committee members}
The following is a non-exhaustive list of potential invitees:
  \begin{multicols}{3}\footnotesize
    \begin{itemize}
        \item Daniel Ahlin - (PDC,KTH)
        \item Tim Brown -(UC-Boulder)
        \item Wolfgang Frings, (JSC)
        \item Andy Georges - (U. Ghent)        
	\item Vera Hansper (CSC-Finland)
        \item Chris Harris - (Pawsey)
        \item Kenneth Hoste - (U. Ghent)
	\item Paul Kolano, (NASA-Ames)
        \item James Lin (Shanghai JTU)
        \item John Linford (ParaTools)
        \item Robert McLay (TACC)
        \item Dave Montoya (LANL)
 	\item Randy Schauer - (ARL)
        \item Dane Skow (Dane Skow Enterprises)       
        \item William Scullin - (ANL)
	\item Karen Tomko - (OSC)
    \end{itemize}
  \end{multicols}



\begin{thebibliography}{3}
% references

\bibitem[Bor14]{H14}{Bording, C. and Georges,A. },``Proceedings of the First International Workshop on HPC User Support Tools," \emph{ SC14 International Conference for High Performance Computing, Networking, Storage and Analysis}, New Orleans, LA, 2014.

\bibitem[Bor15]{H15}{Bording, C., Gamblin, T., and Hansper,V. },``Proceedings of the Second International Workshop on HPC User Support Tools," \emph{ SC15 International Conference for High Performance Computing, Networking, Storage and Analysis}, Austin, TX, 2015.

\bibitem[Bor16]{H16}{Bording, C., Gamblin, T., and Hansper,V. },``Proceedings of the Third International Workshop on HPC User Support Tools," \emph{ SC16 International Conference for High Performance Computing, Networking, Storage and Analysis}, Salt Lake City, UT, 2016.

\end{thebibliography}

\begin{itemize} 
    \item[(1)] \url{https://github.com/hpcugent/easybuild/wiki/SC13-BoF-session}
    \item[(2)] \url{https://sites.google.com/site/smallhpc/home}
    \item[(3)] \url{http://www.isc-events.com/isc13_ap/presentationdetails.php?t=contribution&o=2108&a=select&ra=eventdetails}
    \item[(4)] \url{http://sc13.supercomputing.org/schedule/event_detail.php?evid=bof111}
    \item[(5)] \url{http://hpcugent.github.io/easybuild/fosdem14.html}  
    \item[(6)] \url{http://scalability-llnl.github.io/spack/}  
    \item[(7)] \url{http://www.ieee.org/conferences_events/conferences/publishing/templates.html}   
\end{itemize}

\end{document}
