\documentclass[a4paper,11pt]{article}  
\usepackage{graphicx}
\usepackage{hyperref}


\begin{document}
%\begin{center}
 %   \Large{SC'18 workshop proposal}\\~\\
%    \large{\textbf{5th International Workshop on HPC User Support Tools (HUST-17)}}\\~\\
 %   \normalsize{Ralph C. Bording (IBM Research), Elsa Gonsiorowski (LLNL),\\
 %   Olli-Pekka Lehto(Jump Trading,LLC)}
%\end{center}

\section*{Abstract}
\label{sec:abstract}


Supercomputing centres exist to drive scientific discovery by supporting researchers in computational science fields. 
To improve the productivity of the user and the usability of systems in an environment as complex in a typical HPC centre
they employ specialised support teams and individuals. The broad support effort ranges from basic system admin to 
managing 100's of Petabytes with complex hierarchal data-storage systems, to supporting high performance networks,  
or consulting on advanced math libraries, code optimisation, and managing complex HPC software stacks.  Often, 
support teams struggle to adequately support scientists as HPC environments are extremely complex, and combined 
with additional  complexity of multi-user installations, exotic hardware, and maintaining research software, to 
support 100's or even 1000's of HPC users can be extremely demanding.

The HUST workshop, has been the ideal forum where new and innovative tools such as XALT, SPACK and Easybuild, 
have been widely announced to the broader HPC community that results into creating the communities and special 
interest groups about these tools to support entire BoFs, Workshops and Tutorials on these tools at SC, ISC and 
other HPC conferences     We will continue to provide the necessary forum for system administrators, user support 
team members, tool developers, policy makers and end users. We will provide a forum to discuss support issues and 
we will continue to provide a publication venue for current support developments. Best practices, user support tools, 
and novel ideas to streamline user support efforts at supercomputing centres are in scope for the HUST workshop.
%\paragraph{Topic area:} Administration tools, User support
%\paragraph{Keywords:} user support, tools, best practices, Docker, big data, Hadoop and Spark clusters, machine learning, collaboration, community

\end{document}

\

