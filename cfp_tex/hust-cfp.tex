\documentclass[11pt,a4paper]{article}

%\usepackage[tmargin=1in,bmargin=1in,lmargin=1.25in,rmargin=1.25in]{geometry}
\usepackage[margin=1.25in]{geometry}
\usepackage{graphicx}
\usepackage{multicol}
\usepackage{enumitem}
\usepackage[colorlinks=true,linkcolor=blue,citecolor=blue,urlcolor=blue]{hyperref}
\usepackage[sectionbib,numbers]{natbib}
\usepackage[UKenglish]{isodate}
\usepackage{watermark}
\usepackage{transparent}
\usepackage{fix-cm}
\usepackage{fancyhdr}
%\usepackage[utf8]{inputenc}
%\usepackage{pxfonts}
\usepackage[default,osfigures]{opensans}
\usepackage[T1]{fontenc}

\usepackage{sectsty}
%\subsubsectionfont{\itshape}
%\usepackage{draftwatermark}
%\SetWatermarkText{DRAFT}
%\SetWatermarkScale{1}
%\SetWatermarkLightness{0.95}

\setlist{noitemsep}
\parskip 2pt

\usepackage{titlesec}

\titlespacing\section{0pt}{12pt plus 4pt minus 2pt}{4pt plus 2pt minus 2pt}
\titlespacing\subsection{0pt}{12pt plus 4pt minus 2pt}{2pt plus 2pt minus 2pt}
\titlespacing\subsubsection{0pt}{12pt plus 4pt minus 2pt}{2pt plus 2pt minus 2pt}

\begin{document}

\begin{titlepage}
\begin{center}
\vspace{40mm}
{ {\fontfamily{phv}\selectfont \Huge Call for participation}}\\
\vspace{4mm}
{\bf {\fontfamily{phv}\selectfont \huge HUST-2018:  Fifth International Workshop on HPC User Support Tools.}}\\
\vspace{20mm}

{ {\fontfamily{phv}\selectfont \huge Held in conjunction with: }}\\
\vspace{2mm}
{ {\fontfamily{phv}\selectfont \huge SC18: The International Conference on High
Performance Computing, Networking, Storage and Analysis.}}\\
\vspace{80mm}
{ {\fontfamily{phv}\selectfont \Huge Dallas, TX, USA}}\\
\vspace{2mm}
{ {\fontfamily{phv}\selectfont \Huge  November 2018}}\\
\vspace{35mm}

\includegraphics[width=35mm]{../img/SC18-color-hor.jpg}
\hspace{15mm}
\includegraphics[width=35mm]{../img/tchpc_logo_cmyk.png}
\hspace{15mm}
\includegraphics[width=35mm]{../img/IEEE-eps/ComputerSocietyLogo-RGB-stacked.png}
\\

\end{center}
\end{titlepage}

\lhead{}
\chead{\bf \fontfamily{phv}\selectfont HUST-2018: Fifth International Workshop on HPC User Support Tools.}
\rhead{}
\pagestyle{fancy}

\setcounter{page}{2}

\parindent 0pt
\parskip 6pt
\pagebreak
\section{Introduction}

Supercomputing centers exist to drive scientific discovery by supporting researchers in 
computational science fields.  To make users more productive in the complex HPC
environment, HPC centers employ user support teams.  These teams
serve many roles, from setting up accounts, to consulting on math libraries and code
optimization, to managing HPC software stacks.
Often, support teams struggle to adequately support scientists.
HPC environments are extremely complex, and combined with
the complexity of multi-user installations, exotic hardware, and maintaining
research software, supporting HPC users can be extremely demanding.

With the fifth international HUST workshop, we will continue to provide a necessary forum for 
system administrators, user support team members, tool developers, policy makers and
end users.  We will provide a forum to discuss support issues and we will
provide a publication venue for current support developments.  Best practices,
user support tools, and any ideas to streamline user support at supercomputing
centers are in scope.

\section{Topics}

Topics of interest include, but are not limited to:

\begin{itemize}
\item defining and customising the user environment
\item software build and installation tools
\item tools and frameworks for using system performance analysis tools
\item workflow and pipeline tools
\item collaboration tools
\item novel environments: cloud, support for Docker.
\item supporting Hadoop and Spark clusters for {\bf Big Data}
\item establishing baseline configuration efforts for HPC
\item software tools for system {\bf testing and monitoring}
\item documentation: {\bf creating, maintaining and auto-updating}
\end{itemize}

\section{Submission Details}

We invite authors to submit original, high-quality work with
sufficient background material to be clear to the HPC
community. 
\subsection{Format}
Submissions are at a length of 8 pages(IEEE conference format) with a 10-
page limit (see https://www.ieee.org/conferences/publishing/templates.html). 
The page limit includes figures, tables, and your appendices, 
but does not include references, for which there is no page limit. 
Papers should be submitted in PDF format. We kindly refer authors to 
the necessary templates [2].

All submissions should be made electronically through the SC18 submissions 
website at https://submissions.supercomputing.org.  Submissions must be 
double blind, i.e., authors should remove their names, institutions or hints 
found in references to earlier work. When discussing past work, they need to refer to
themselves in the third person, as if they were discussing another
researcher's work. Furthermore, authors must identify any conflict of
interest with the PC chair or PC members.

Proceedings will be published in the IEEE Xplore digital
library through collaboration with IEEE TCHPC.

\section{Important dates}
\begin{itemize}
\item Submission deadline EXTENED: September 10th 2018
\item Workshop paper reviews: by September 21st 2018
\item Acceptance notifications: October 2nd 2018
\item Camera-ready papers: October 12th 2018
\item Workshop: At SC'18, November 11th 2018
\end{itemize}

\section{Organisers}

\begin{itemize}
\item Christopher Bording, IBM Research, United Kingdom 
\item Elsa Gonsiorowski, LLNL, USA
\item Olli-Pekka Lehto, Jump Trading, LLC, Singapore
\end{itemize}

\subsection{General Chair}
\begin{itemize}
\item Christopher Bording
\end{itemize}
\subsection{Program Committee Co-Chairs}

\begin{itemize}
\item Elsa Gonsiorowski
\item Olli-Pekka Lehto
\end{itemize}

\subsection{Program Committee}

\begin{itemize}
\item Daniel Ahlin, PDC HPC Center, KTH Royal Institute of Technology, Sweden
\item David E. Bernholdt, Oak Ridge National Laboratory, USA
\item Erik Engquist, Rice University, USA
\item Fabrice Cantos, National Institute of Water and Atomospheric Research, New Zealand
%\item Markus Griemer, Juelich Supercomputing Centre, Germany
\item Christopher Harris, Pawsey Supercomputing Centre, Australia
\item Paul Kolano, NASA, USA
\item John C. Linford, ARM, USA
\item Robert McLay, Texas Advanced Computing Center, USA
\item Dave Montoya, Los Alamos National Laboratory, USA
\item Randy Schauer, Raytheon Company, USA
%\item Gary B. Skouson, Pacific Northwest Nation Laboratory, USA
%\item William Scullin, Argonne National Laboratory, USA
\item Alan D Simpson, Edinburgh Parallel Computing Centre, United Kingdom
\item Karen Tomko, Ohio Supercomputing Center, USA
\item Jianwen Wei, Shanghai Jiao Tong University, China
\end{itemize}

\end{document}
